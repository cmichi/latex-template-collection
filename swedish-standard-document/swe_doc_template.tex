% My template for s standard swedish document with custom set margins.
% Feel free to use this template for your own documents.
% The text in the template is a short version on the three little pigs
% Jack-Benny Persson.

\documentclass[11pt,a4paper,swedish]{article}
\usepackage[swedish]{babel}
\usepackage[utf8]{inputenc}
\usepackage[a4paper]{geometry}
\geometry{verbose,tmargin=2.5cm,bmargin=2.5cm,lmargin=2.5cm,rmargin=2.5cm}

\title{Min \LaTeX{} mall för ett vanlig dokument}
\author{Jack-Benny Persson}
\date{April, 2012}

\begin{document}

\maketitle
\pagebreak

\tableofcontents

\pagebreak

\section{De tre små grisarna}

Det var en gång tre små grisar med fina knorrar som bodde i skogen tillsammans med sin mamma. En dag sa hon till dem:

\textit{''Nu har ni blivit stora nog att flytta hemifrån. Ge er nu ut och bygg var sitt hus åt er. Men det måste vara ett rejält hus, för den elaka vargen ligger på lur, och han tar er om han kan.''}

De tre små grisarna packade sina saker och gav sig iväg.
Den första lilla grisen byggde sig ett hus av halm. Han brydde sig inte om att vara rädd för vagen. Den andra lille grisen byggde sig ett hus av trä, han var inte heller särskilt rädd för vargen. Den tredje lille grisen byggde ett hus av stenar och murbruk. Han lade på ett tak av tegel och satte i en dörr av tjockt trä. Hans hus blev verkligen stabilt. Han kände sig mycket nöjd med sig själv när han var klar.

\subsection{Vad hände sen?}

Vargen fick höra talas om att de tre små grisarna flyttat hemifrån. En dag kom han för att äta upp dem. De tre små grisarna blev mycket rädda och skyndade in i sina hus.
Först knackade vargen på dörren till huset av halm.

\section{Tack till}

Tack till sagornas värld där allt kan hända.

\end{document}
